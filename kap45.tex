\subsection{Feldquantisierung der Diractheorie}
\subsubsection{Erzeugungs- und Vernichtungsoperatoren}
Einteilchenlösungen:
\begin{eqnarray*}
\psi^\pm_{\sigma k}(x) \g \sqrt{\frac{m}{VE_k}} u_\sigma^\pm(k) \ e^{\mp i k x}\\
\text{mit  }u_\sigma^+(k) \g u_\sigma(k) = \sqrt{\frac{E+m}{2 m}} \left(\begin{array}c \chi_\sigma\\\frac{\vec \sigma \vec k}{E+m}\chi_\sigma\end{array}\ri)\\
u_\sigma^-(k) \g v_\sigma(k) = \sqrt{\frac{E+m}{2 m}} \left(\begin{array}c \frac{\vec \sigma \vec k}{E+m} \chi_\sigma\\\chi_\sigma\end{array}\ri)
\end{eqnarray*}
Es gilt mit adjungierten Spinoren
\begin{eqnarray*} \bar u_\sigma^\alpha \g u_\sigma^{\alpha \dagger}\gamma^0\qquad \alpha = \pm\\
\bar u_\sigma^\alpha u_{\sigma'}^\beta \g\alpha \delta_{\alpha \beta }\delta_{\sigma\sigma'}
\end{eqnarray*}

\underline{Definition}
\begin{center}
	\begin{tabular}{|l l|l|}\hline
		$b_{\sigma k}^\dagger$& erzeugt & Elektronen \\ \cline{1-2}
		$b_{\sigma k}$ & vernichtet & \\\hline
		$d_{\sigma k}^\dagger$ & erzeugt & Positronen \\ \cline{1-2}
		$d_{\sigma k}$ & vernichtet&\\\hline
	\end{tabular}
\end{center}
Spinor $u^\pm_\sigma (\vec k)$ mit Energie $ E = k^0 = \sqrt{k^2 + m^2} > 0 $ und Spin  $ \sigma$ . 
Da Elektronen/Positronen Fermionen sind, führen wir Antivertauschungsrelationen ein:
\begin{eqnarray*}
\left \{b_{\sigma k}, b^+ _{\sigma'k'}\ri \} = \delta_{\sigma\sigma'}\delta_{k k'}% = \left \{\delta_{\delta k}, \delta^\dagger_{\sigma' k'}\ri\}
\qquad \qquad \{d_{\sigma k},d_{\sigma' k'}^+ \} = \delta_{\sigma \sigma'}\delta_{k k'}
\end{eqnarray*}
alle anderen verschwinden.
\subsubsection{Feldoperatoren}
Wir entwickeln die Feldoperatoren nach einem \underline{vollständigen} Satz von Lösungen.
\begin{eqnarray*}
\wh\Psi(x) = \sum_{\sigma k } \underbrace{\wh\Psi^+_{\sigma k}(x)b_{\sigma k}}_{\text{nur für Elektronen}} + \underbrace{\wh\Psi^-_{\sigma k}(x) d_{\alpha k }^\dagger(x) }_{\text{wegen Vollständigkeit}}
\end{eqnarray*}
Der adjungierte Feldoperator
\begin{eqnarray*}
\bar{\wh\Psi}(x) = \wh\Psi^\dagger(x) \gamma^0 = \sum_{\sigma k}\bar{\wh\Psi}^+_{\sigma k}(x) b_{\sigma k}^\dagger + \bar{\wh\Psi}_{\sigma k}^-(x) d_{\sigma k}
\end{eqnarray*}
Die Feldoperatoren erfüllen komplizierte Vertauschungsrelationen, da die Kausalität (welche sich mit Lichtgeschwindigkeit ausbreitet) gewahrt bleiben muss. Man findet jedoch für gleiche Zeiten (also beispielsweise im nichtrelativistischen Grenzfall) folgende Vertauschungsrelationen: 
\begin{eqnarray*}
\left\{\wh\Psi_\alpha(t,\vec x), \wh\Psi_\beta^\dagger(t,\vec x)\ri\} = \delta_{\alpha\beta}\delta(\vec x-\vec x')
\end{eqnarray*}
d.h. $\wh\Psi$ und $\wh\Psi$ erfüllen die für fermionische Feldoperatoren geforderten Vertauschungsrelationen.

\underline{Spin-Statistik-Theorem} 
Teilchen mit ganzzahligem/halbzahligem Spin sind Bosonen/Fermionen. Ursache ist, dass die entsprechenden Feldoperatoren für raumartige Abstände vertauschen/antivertauschen müssen.

\subsubsection{Hamiltonoperator}
Der Vierer-Impulsoperator ist
\begin{eqnarray*}
\hat p^\mu &=& \hbar \int d^3x \ \bar{\wh\Psi}(x) (i\dell^\mu \gamma_0) \wh\Psi(x)\\
&\vdots&\\
&=& \hbar\sum_{ k \sigma} k^\mu \left( \hat b^\dagger_{\sigma k }\hat b_{\sigma k} - \hat d_{\sigma k}\hat d^\dagger _{\sigma k}\ri ).
\end{eqnarray*}
Den Hamiltonoperator erhalten wir als die 0-Komponente des Vierer-Impulsoperators. 
\begin{eqnarray*}
\wh\ham &=& \hat p^0 = \sum_{k \sigma} E_{k}\left(\hat b^\dagger_{\sigma k} \hat b_{\sigma k} - \hat d_{\sigma k}\hat d^\dagger _{\sigma k}\ri)
\\
&=& \sum_{k\sigma} E_{k}\big(\underbrace{\hat b^\dagger_{\sigma k }\hat b_{\sigma k}}_{=\hat n^+_{\sigma k}} + \underbrace{\hat d_{\sigma k }^\dagger\hat d_{\sigma k}}_{=\hat n^-_{\sigma k}} -\1 \big) \qquad  \text{mit: } E_k=\hbar k^0=\sqrt{m^2c^4+c^2\hbar^2 \vec{k}^2}
\end{eqnarray*}
Für die Umformung wurde dabei die Antivertauschungsrelation: $\hat d \hat d^\dagger = \1 -\hat d^\dagger\hat d$

Die $\hat n_{k\sigma}$ sind Teilchenzahloperatoren für den Zustand $k\sigma$. Es gilt: $(\hat n_{\sigma k}^\alpha)^2 = \hat n_{\sigma k}^\alpha$, damit hat $n_{\sigma k}^\alpha$ die Eigenwerte 0 und 1. 

Damit ist die Energie immer positiv, da sowohl die Elektronen als auch die Positronen einen positiven Energiebeitrag liefern.

Der letzte Term liefert eine (unendliche) Nullpunktsenergie selbst im Vakuumszustand.

Wir subtrahieren diesen Term vom Hamiltonoperator, um zur physikalischen Observablen Energie zu kommen.
\begin{eqnarray*}
\wh\ham_{\text{neu}} \g \sum \limits_{\sigma k} E_{k} \left (\hat b_{\sigma k}^\dagger \hat b_{\sigma k}+\hat d_{\sigma k}^\dagger\hat d_{\sigma k}\ri)\\
\hat\vec p_{\text{neu}} \g \sum \limits_{\sigma k}\hbar \vec k \left(\hat b_{\sigma k}^\dagger \hat b_{\sigma k} + \hat d_{\sigma k}^\dagger \hat d_{\sigma k}\ri)
\end{eqnarray*}
Der Ladungoperator $\hat Q$ entspricht der 0-Komponte der Viererstromdichte $\hat j^\mu$
\begin{eqnarray*}
\hat Q &=& -e \int \mathrm{d}^3x\; \bar{\wh\Psi}\gamma_0\wh\Psi = j^0 = \dots = -e \sum \limits_{\sigma k}\left(\hat b_{\sigma k}^\dagger \hat b_{\sigma k}-\hat d_{\sigma k}^\dagger \hat d_{\sigma k}\ri) 
\\
\Rightarrow\quad \langle\hat Q\rangle &=& \bra{\psi}{\hat Q}\ket{\psi} =  \Bigg{\{}\!\!\begin{array}{ll} \bra{0}\hat b_{k\sigma} \;\hat Q\; \hat b_{k\sigma}^\dagger\ket{0} & = -e  \\ \bra{0}\hat d_{k\sigma} \;\hat Q\; \hat d_{k\sigma}^\dagger\ket{0} & = +e  \end{array}
\end{eqnarray*}
Damit erzeugt $\hat b^\dagger_{k\sigma}$ Elektronen mit negativer Ladung $-e$ und $\hat d^\dagger_{k\sigma}$ Positronen mit positiver Ladung $+e$. 




