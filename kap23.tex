\subsection{Lagrangeformalismus}
\subsubsection{Mechanik eines freien Teilchens}
Die Wirkung ist $S = \int \Lr(q,\dot q,t)dt$.\\
\underline{Problem:} $dt$ ist nicht kovariant.\\
\underline{Lösung:} Wir führen die Eigenzeit $d\tau = \frac{dt}{\gamma}$ ein.
\begin{eqnarray*} \longrightarrow S_{\text{frei}} =\int \gamma \Lr_{\text {frei}}d\tau \end{eqnarray*}
Damit das Prinzip der kleinsten Wirkung gilt, muss $S$ (und damit $\delta S = 0$) ein Skalar sein! Da ein freies Teilchen orts- und richtungsunabhängig ist, ist der einzige geschwindigkeitsabhängige Skalar der Geschwindigkeitsbetrag:
\begin{eqnarray*} \text{mit  } u_{\mu}u^{\mu} = c^2\\ \Longrightarrow S \g - \int mc^2d\tau \\
\text{bzw. } S \g \int - \underbrace{m c^2 \sqrt{1 - \frac{\vec v ^2}{c^2}}}_{\Lr_{\text{frei}}=\sqrt{u_{\mu}u^{\mu}}mc}dt\end{eqnarray*}
Beachte: $u_{\mu}u^{\mu}$ darf erst nach Variation verwendet werden, da $u^0(\tau)=\frac{\text{d}t}{\text{d}\tau}$ erst als unabh‰ngige Funktion ausgerechnet werden muss.
Dies ergibt:
\begin{eqnarray*}
\frac d{dt} \frac{\partial \Lr}{\partial v_i} = \frac d {dt} m \gamma \vec v = 0 \qquad \widehat = \text{ konst. Geschwindigkeit}\end{eqnarray*}
\subsubsection{Mechanik mit elektromagnetischem Feld}
\begin{eqnarray*} \Lr = \Lr_{\text{frei}} + \Lr_{\text{int}} \qquad \text{mit  }\gamma\Lr \wh = \text{  Skalar}\end{eqnarray*}
Forderungen an $\Lr_{\text{int}}$ (neben der Kovarianz)
\begin{itemize}
\item Linear in der Ladung $q$
\item Linear in dem Potential $A^{\mu}$
\item Translationsinvariant ( explizit unabhängig von $x^{\mu}$ )
\item Möglichst Funktionen 1. Ableitung
\end{itemize}
\begin{eqnarray*}\longrightarrow \Lr_{\text{int}} \g - q u_{\mu} \frac{A^{\mu}}{\gamma} \\
\g - q \phi + q \vec v\vec A\\
\end{eqnarray*}
\begin{eqnarray*}
\Longrightarrow \boxed{ \Lr = -mc^2 \sqrt{1 - \frac{\vec v^2}{c^2}} + q \vec v\vec A - q \phi}
\end{eqnarray*}
\underline{Kanonische Impulse:}
\begin{eqnarray*}
\vec p_{\text{kan}} = \frac{\partial \Lr}{\partial \vec v} = m\gamma\vec v + q \vec A\\
\Longrightarrow \vec v = c ^2\frac{\vec p - q \vec A}{\sqrt{m^2c^4 + c^2(\vec p - q\vec A)^2}}
\end{eqnarray*}
\underline{Hamiltonfunktion:}
\begin{eqnarray*}
\ham = \sqrt{m^2c^4 + c^2(\vec p-q\vec A)^2} + e\phi \end{eqnarray*}
\underline{Bemerkung:} \babsatz Der Hamiltonformalismus kann nicht mehr mit der Energie identifiziert werden, da die Hamiltonfunktion nun ein Lorentzskalar ist und die Energie die Zeit-Komponente eines Vierervektors sein sollte.\eabsatz \vspace{0.5cm}

\underline{Kovariante Form der Lagrangefunktion}\\
Mit Hilfe der Vierer-Geschwindigkeit 
\begin{eqnarray*}
\sqrt{1 - \frac{\vec v^ 2}{c^ 2}} = \frac 1 { \gamma} \frac{u_{\mu} u^ {\mu}}{c^2}
\end{eqnarray*}
und der Eigenzeit $d\tau = \gamma dt$ lässt sich die Wirkung umschreiben:
\begin{eqnarray*}
S_{\text {frei}} = -mc \int \sqrt{u_{\mu}u^ {\mu }} d\tau\\
\text{mit Zwangsbedingung}\qquad u_{\mu}u^ {\mu} = c^ 2\\
\Longrightarrow \quad S_{\text{frei}} = \text{ const.}
\end{eqnarray*}
Ausweg: \babsatz  Wir führen ein Bahnparameter ($s$) ein, der \underline{nach} der Variation nach $ds = cd\tau$ mit der Eigenzeit identifiziert wird.\\
(D.h. $ \frac{dx^ {\mu}}{ds}\frac {dx_{\mu} }{ds} \neq 1$.)\eabsatz
\begin{eqnarray*}
\Longrightarrow S_{\text{frei}} = -mc \int \sqrt{g_{\mu\nu}\frac{dx^ {\mu}}{ds} \frac{dx^{\nu} }{ds} }ds\\
\text{Für die Ww. Mit dem e.m-Feld gilt:}\\
S_{\text{int}} = - q \int \frac{dx^ {\mu}}{ds} A_{\mu}(x) ds
\end{eqnarray*}
die Variation von $S_{\text{frei}} + S_{\text{int}}$ mit $ds = cd\tau$ und $u_{\mu}u^ {\mu} = c^ 2$ ergibt:
\begin{eqnarray*}
\Longrightarrow&{}& m \frac{d^ 2x^ {\mu}}{d\tau^ 2} + q
\underbrace{\frac{dA^ {\mu}}{d\tau}}_{=
  \frac{dx_{\nu}}{d\tau}\partial^ {\nu}A^ {\mu}} - q\frac{dx_{\nu}}{d\tau} \partial^ {\mu} A^ {\nu} = 0\\ 
\text{mit }&{}& \quad \big ( \partial^  {\nu}A^ {\mu} - \partial^ {\mu}A^ {\nu}\big ) = F^ {\nu\mu}\\
\g m \frac{d^ 2x^ {\mu}}{d\tau^ 2} + q \frac{dx_{\nu}}{d \tau} F^ {\nu\mu} = 0
\end{eqnarray*}
\subsubsection{Felder}
Behandlung von Feldern im Lagrangeformalismus: Fasse die Felder $\phi_k(x)$ als ,,k-te Koordinate eines Freiheitsgrades bei $x^\mu$ auf. Hierbei ist $x$ der Viererortsvektor.\\
In 4-Dimensionen ergibt dies:



\begin{center}
\begin{tabular}{ccc|c}
 & &Hier&Klassisch\\
Kontinuierlicher Index/Indizes&$\longrightarrow$&$x^ {\mu}$&$t$\\
Auslenkung&$\longrightarrow$&$\phi_k(x)$&$x(t)$\\
Geschwindigkeit&$\longrightarrow$&$\partial^ {\mu}\phi_k(x)$&$\partial_t x(t)$
\end{tabular}
\end{center}

Die Wirkung ist
\begin{eqnarray*}
S = \int d^ 4x \Lr\big(\phi_k,\partial ^ {\mu}\phi_k\big)
\end{eqnarray*}
Die Euler-Lagrangegleichung ergibt sich zu:
\begin{eqnarray*}
\partial^ {\mu} \frac{\partial \Lr}{\partial ( \partial^ {\mu}\phi_k)} - \frac{\partial \Lr}{\partial \phi_k} = 0
\end{eqnarray*}
\begin{itemize}
\item $k$ partielle Differentialgleichungen für $\phi_k(x)$.
\item $\Lr \wh =$Lagrangedichte.
\item $dx^ 4$ ist ein Lorentzskalar $\longrightarrow$ $\Lr$ muss ebenfalls ein Lorentzskalar sein.
\end{itemize}
Anwendung auf die Elektrodynamik:
\begin{itemize}
\item Freie Felder (Felder ohne Quellen) betrachten vier Potentiale ($A^ {\mu}$)\\
$\longrightarrow \ \Lr$ hängt nur von den Ableitungen ab.
\item Die Kopplung an der Materie ist linear in $A^ {\mu}$.
\end{itemize}
Ansatz:
\begin{eqnarray*} \boxed{ \Lr = \frac {-1}{4\mu_0} F_{\mu\nu}F^ {\mu\nu} - j_{\mu} A^ {\mu}}\end{eqnarray*}
Es gilt die Euler-Lagrangegleichung:
\begin{eqnarray*} \frac{\partial \Lr}{\partial A^ {\mu}}\g - j_{\mu}\\
\rightarrow -4\mu_0 \frac{\partial \Lr}{\partial\big ( \partial^ {\nu}A^{\mu}\big)} \g \frac{\partial}{\partial \big(\partial^ {\nu}A^{ \mu}\big)} g_{\alpha\beta}g_{\gamma\delta}F^ {\alpha\gamma}F^{\beta\delta}\\
\g g_{\alpha\beta}g_{\gamma\delta}\frac{\partial}{\partial\big(\partial^{\nu}A^{\mu}\big)} \Bigg[\Big( \underbrace{\partial^{\alpha}A^{\gamma}-\partial^{\gamma}A^{\alpha}}_{=F^{\alpha\gamma}}\Big)\Big(\underbrace{\partial^ {\beta}A^ {\delta}-\partial^ { \delta}A^ {\beta}}_{=F^{\beta\delta}}\Big)\Bigg]\\
\text{mit } \frac{\partial \big (\partial^ {\alpha}A^ {\gamma}\big )}{\partial\big(\partial^ {\nu}A^ {\mu}\big )} = \delta_{\nu}^ {\alpha}\delta_{\mu}^ {\gamma}\\
\g g_{\alpha\beta}g_{\gamma\delta} \Bigg[\Big(\delta_{\nu}^ {\alpha}\delta_{\mu}^ {\gamma}-\delta_{\nu}^ {\gamma}\delta_{\mu}^ {\alpha}\Big) F^ {\beta\delta} + \Big(\delta_{\nu}^ {\delta}\delta_{\mu}^ {\delta} - \delta_{\nu}^ {\delta}\delta_{\mu}^ {\beta}\Big )F^ {\alpha\gamma} \Bigg]\\
\g F_{\nu\mu}-F_{\mu\nu}+F_{\nu\mu}-F_{\mu\nu} = 4 F_{\nu\mu}\\
\Longrightarrow &{}& \boxed{\partial^ {\mu}F_{\mu\nu} = \mu_0 j_{\nu}}
\end{eqnarray*}
Die obige Gleichung ist die inhomogene Maxwellgleichung.\\
Die homogenen Maxwellgleichungen sind durch den Ansatz $F^ {\mu\nu} = \partial^ {\mu}A^ {\nu}-\partial^ {\nu}A^ {\mu}$ automatisch erfüllt.\\
Es gilt ebenfalls:
\begin{eqnarray*}
\underbrace{\partial^{\nu}\partial^{\mu}F_{\mu\nu}}_{\partial^ {\nu}\partial^ {\mu}=\partial^ {\mu}\partial^ {\nu} \text{ aber }F^ {\mu\nu}= -F^ {\nu\mu}} = -\mu_0\partial^ {\nu}j_{\nu}
\end{eqnarray*}
$\Longrightarrow \partial^ {\nu} j_{\nu} = 0$ Die Kontinuitätsgleichung ist erfüllt.\\


\subsubsection{Felder und Teilchen}
Für ein Teilchen $i$ mit \{$x_{(i)}^{\mu}(\tau)$\} und Felder $F^ {\mu\nu}(x^ {\mu})$ gibt es eine Lagrangefunktion mit dem Wechselwirkungsterm
\begin{eqnarray*} \int ds \ q \frac{dx^ {\mu}} {d s}\ A_{\mu}\left(x(s)\ri) \leftrightarrow \int d^4x \ j_{\mu}(x)A^ {\mu}(x).\end{eqnarray*}
Die Stromdichte für ein Teilchen ist
\begin{eqnarray*} j^ {\mu}(x) = \sum_i\int ds \ q_i \frac{dx^ {\mu}}{d s}\delta^ {(4)}(x-x_i(s)). \end{eqnarray*}
Einsetzen ergibt den obigen Wechselwirkungsterm, wobei $s$ mit $\tau$ identifiziert werden kann.
Ebenso ist die Lagrangefunktion für ein freies Teilchen
\begin{eqnarray*}
\Lr_{\text{frei}}\left( \frac{dx}{ds}\ri) = \int d^ 4x' \Lr \left( \frac{dx'}{ds}\ri )\delta^ {(4)}(x'-x(s)).\end{eqnarray*}
Die gesamte relativistische Mechanik und Elektrodynamik ist bestimmt durch:
\begin{eqnarray*}
S \g \int d^4x \ \Lr \left ( \Big\{x_i, \partial^{\mu}x_i\Big \} ,A^{\mu}(x),\partial^{\mu} A^{\nu}(x) \ri )\\
\Lr \g \sum \limits_i - m_ic \int d\tau \delta^{(4)}(x-x_i(\tau))\sqrt{u_{i}^{\mu}u_{i\;\mu}}\\ &{}& - \sum\limits_i q_i \int d\tau \frac{dx_{i}^{\mu}(\tau)}{d\tau} \delta^{(4)}\big(x-x_{i}(\tau)\big)A_{\mu}(x) \\&{}& -\frac 1 {4 \mu_0}F_{\mu\nu}F^{\mu\nu}
\end{eqnarray*}
\begin{itemize}
\item Variation von $S$ nach $x_i(s_i)$ und $A_{\mu}(x)$ ergibt die Bewegungsgleichung der Teilchen und die Maxwellgleichungen.
\item Die Beschreibung der effektiven Wechselwirkung zwischen den Teilchen wird nicht einfach sein, da die Potentiale von dem Ort der Teilchen zu retardierten Zeiten abhängen. $\longrightarrow$ nicht instantane Wechselwirkung.
\item In niedrigster relativistischer Ordnung $\longrightarrow$ Darwin Terme (siehe Übung).
\end{itemize}
\subsubsection{Energie-Impuls- oder Spannungstensor}
Für eine allgemeine Feldtheorie gilt:
\begin{eqnarray*}
\Lr(\varphi,\partial^{\mu}\varphi).
\end{eqnarray*}
$\Lr$ ist die {\bf Lagrangedichte} und ist explizit unabhängig von $x^{\mu}$.\\
Die Euler-Lagrangegleichung ist somit:
\begin{eqnarray*}
\partial_{\mu} \frac{\partial \Lr}{\partial(\partial_{\mu} \varphi)} - \frac{\partial \Lr}{\partial \varphi} = 0
\end{eqnarray*}
Wir betrachten die Erhaltungsgrößen:
\begin{eqnarray*}\nonumber
\partial _{\mu} \Lr \g \big(\partial _{\mu}\varphi\big) \frac{\partial \Lr}{\partial \varphi} + \big(\partial_{\mu}\partial_{\nu}\varphi\big) \frac{\partial \Lr}{\partial(\partial_{\nu} \varphi)}\\ \nonumber
\text{mit } &{}&\quad \partial_{\nu} \frac{\partial \Lr}{\partial(\partial_{\nu}\varphi)} = \frac{ \partial \Lr}{\partial \varphi}\quad \text{Euler-Lagrange}\\ \nonumber
\longrightarrow 0 \g \partial_{ \nu } \underbrace{\left((\partial _{\mu}\varphi)\frac{\partial \Lr}{\partial(\partial_{\nu} \varphi)}\ri)}_{T_{\ \ \mu}^{\nu}- \delta_{\mu}^{\ \ \nu}}\\ \label{eq:4}
\Longrightarrow &{}&\boxed{\partial_{\nu} T_{\ \ \mu}^{\nu} = 0}
\end{eqnarray*}
$T_{\ \ \mu}^{\nu}$ ist somit eine kovariante Erhaltungsgröße analog zur Energie-Impulserhaltung in der klassischen Mechanik.\vspace{1.5cm}



\underline{Physikalische Bedeutung von  $T^ {\mu\nu}$}\\ \\
Die Erhaltung ist eine Folge der Translationsinvarianz (unabhängig von $x^ {\mu}$) und gilt zu allen Zeiten und im ganzen Raum.
\begin{eqnarray*}
T^ {00} = \dot \varphi \frac{\partial \Lr}{\partial \dot \varphi} - \Lr
\end{eqnarray*}
Diese Gleichung entspricht genau der {\bf Energiedichte} $W$ des Feldes (analog einer Hamiltondichte).\\
Aus der Ableitung Gleichung (\ref{eq:4}) folgt:
\begin{eqnarray*}
\partial_{\nu} T^ {\nu 0} = \frac 1 c  \partial_t T^ {00} + \partial_iT^ {i0} = 0
\end{eqnarray*}
Dies ist die Kontinuitätsgleichung für die Energiedichte.
\begin{eqnarray*} \Longrightarrow T^ {i 0 } = \frac 1 c S^i \qquad \qquad \wh = \text{     \bf Energiestromdichte}\end{eqnarray*}
Integriere Gleichung (\ref{eq:4}) über das 3 dimensionale Volumen:
\begin{eqnarray*} \partial_0 \int T^ {0\mu}d^ 3x + \underbrace{\int \partial_k T^ {k\mu}d^3x}_{= 0,\ \ \  *}= 0\\
\Longrightarrow P^ {\mu} = \int T^ {0\mu}d^ 3x \end{eqnarray*}
ist zeitlich erhalten und entspricht dem Viererimpuls des Feldes.\\
Zu $*$) Diese Beziehung gilt nur für lokalisierte Felder, hier ist nur ein endlicher Raum betroffen und der Satz von Gauss ist über das Volumen des Raumes anwendbar.\\ \\
\underline{Achtung:} \babsatz $T^ {\mu\nu}$ ist i.Allg. nicht symmetrisch definiert.\\
Kann aber symmetrisiert werden, ohne dass sich die Erhaltungssätze ändern.\\
$\longrightarrow$ Die symmetrische Wahl wird wegen der Drehinvarianz bevorzugt.\eabsatz
Die weiteren Komponenten werden in der {\bf Impulsstromdichte} $T^ {ik}$ zusammengefasst:\\
z.B.
\begin{eqnarray*}\label{eq:6} \frac 1 c\frac{\partial}{\partial t} T^ {i 0} + \frac {\partial}{\partial x^ j}T^ {i j} = 0 \end{eqnarray*}
Die Gleichung (\ref{eq:6}) hat die Form der Kontinuitätsgleichung einer Impulsdichte $T^ {i 0}$\\
,,Anschaulich'' : BILD\\
Konkret : Energie-Impuls-Tensor (symmetrisiert).
\begin{eqnarray*}
T^ {\mu\nu} = \left ( \begin{array}{cccc} W & \frac{S_x}{c}&\frac{S_y}c&\frac{S_z}c\\
\frac{S_x}c&T_{xx}&T_{xy}&T_{xz}\\ \frac{S_y}c&T_{xy}&T_{yy}&T_{yz}\\
\frac{S_z}c&T_{xz}&T_{yz}&T_{zz} \end{array} \ri)
\end{eqnarray*}

\subsubsection{Energie-Impuls-Tensor des elektromagnetischen Feldes}
\begin{eqnarray*} 
\Lr \g \frac {- 1} {4\mu_0} F_{\mu\nu}F^ {\mu\nu} \qquad \text{mit} F_{\mu\nu} = \partial_{\mu}A_{\nu} - \partial_{\nu}A_{\mu}\\
T^ {\mu\nu } \g \frac{\partial A_{\kappa}}{\partial x_{\mu }} \cdot \frac{\partial \Lr}{\partial \big(\frac{\partial A_{\kappa}}{\partial x_{\nu}}\big )} - g^ {\mu\nu} \Lr\\
\g - \frac 1 {\mu_0} \big(\partial^ {\mu}A^ {\kappa}\big)F^ {\nu}_{\ \ \kappa} + \frac 1 {4 \mu_0}g^ {\mu\nu}F_{\kappa\lambda}F^ {\kappa\lambda}
\end{eqnarray*}
Wir symmetrisieren durch die Addition von $\frac 1 {\mu_0} \big( \partial^\mu A^\kappa\big ) F^\mu_{\ \ \kappa}$. (Dies ist möglich, da $\partial_{\mu}F^ {\mu}_{\ \ \kappa} = 0$, wenn keine Ladungen existieren.)
\begin{eqnarray*}
\Longrightarrow \boxed{T^ {\mu\nu} = -\frac1 {\mu_0}g_{\kappa \lambda }F^ {\mu\kappa}F^{\lambda \nu} + \frac 1 {4 \mu_0} g^ {\mu\nu}F_{\kappa\lambda}F^ {\kappa\lambda} }\end{eqnarray*}
Eigenschaften von $T^ {\mu\nu}$:
\begin{itemize}
\item symmetrisch $T^{\mu \nu} = T^{\nu\mu}$
\item spurfrei $T_{\mu}^ {\mu}= 0$
\end{itemize}
Die Komponenten ergeben sich zu
\begin{eqnarray*}
T^ {00} \g \frac 1 {2\mu_0} \Big ( \frac 1 {c^ 2} \vec E ^ 2 + \vec B ^ 2\Big) = \frac{\varepsilon_0}2 \vec E + \frac 1 {2 \mu_0} \vec B ^ 2\\ &{}& \quad \longrightarrow \text{\bf Energiedichte}\\
T^{i0}=T^ {0i} \g + \frac 1 {\mu_0 c} \big ( \vec E\times \vec B) _i \\
&{}& \quad  \wh = \text{\bf Poyntingvektor}\\
T^ {ij} \g -\varepsilon_0  E_iE_j - \frac 1 {\mu_0}B_iB_j + \frac{\delta_{ij}}2 T^ {00} \\  &{}& \quad  \longrightarrow\text{\bf negativer Maxwellscher Spannungstensor}
\end{eqnarray*}
Die Berücksichtigung von Ladungen ( $=$ Quellen des elektromagnetischen Feldes) führt auf gemeinsame Erhaltungssätze:\\
z.B.:
\begin{eqnarray*} \frac d{dt} \Big( P^{\mu}_{\text{Feld}} + P_{\text{Teilchen}}^ {\mu}\Big ) \g 0\\
\text{mit } P^ {\mu}_{\text{Teilchen}} = \sum  \limits_i p_{(i)}^ {\mu} &\wh=& \text {gesamter 4er-Impuls aller Teilchen}
\end{eqnarray*}


äää Local Variables: 
äää mode: latex
äää TeX-master: "main"
äää End: 
