\subsection{Quantisierung des Strahlungsfeldes}
\subsubsection{Normalmoden, Photonen}
Lagrangefunktion des freien Feldes
\begin{eqnarray*}\Lr \g -\frac 1 {4\mu_0} F_{\mu\nu}F^{\mu\nu}\qquad\qquad F^{\mu\nu} = \partial^{\mu}A^{\nu}- \partial^{\nu}A^{\mu}\\
\g \frac 1 {2 \mu_0}\left(\frac 1 {c^2}\vec E^2 - \vec B^2\ri) = \frac 1 {2 \mu_0}\left(\frac 1 {c^2}\dot{\vec A}^2- (\rot\vec A)^2 \ri)
\end{eqnarray*}
Die Coulombeichung $\dell_i A^i = 0$ kann durch $A^0=0$ gelöst werden. Es ergibt sich folgende Hamiltonfunktion: 
\begin{eqnarray*}
\ham \g -\frac{\partial \Lr}{\partial \dot A^i}\dot A^i-\Lr =\frac 1 {2\mu_0}\left(\frac 1 {c^2}\vec E^2+\vec B^2\ri)
\end{eqnarray*}
Freie Lösungen bestimmt durch Wellengleichung
\begin{eqnarray*}
	\Box \vec A = 0\qquad (\text{ mit }\divergenz \vec A = 0) 
\end{eqnarray*}
Allgemeine Lösung: Zerlegung nach Normalmoden
\begin{eqnarray*}
A^\mu \g \sum \limits_{\vec k,\lambda=1 ,2} \frac 1 {\sqrt{2 |\vec k|V}}\left(e^{-ikx} \varepsilon ^\mu_{\vec k,\lambda} a_{\vec k,\lambda} + e^{ikx}\varepsilon^{\mu *}_{\vec k,\lambda} a^\dagger_{\vec k,\lambda}\ri)\\
\text{mit }&{}&k_0 = |\vec k|\qquad \vec k\vec \varepsilon_{\vec k,\lambda} = 0\qquad \varepsilon^0_{\vec k,\lambda} = 0\\
&{}&\vec \varepsilon_{\vec k,\lambda} \cdot \vec\varepsilon_{k,\lambda'} = \delta_{\lambda,\lambda'}
\end{eqnarray*}
Wir quantisieren:\\
$a_{\vec k\lambda}$ vernichtet Photon mit $\vec k, \lambda$\\
$a_{\vec k\lambda}^\dagger$ erzeugt Photon mit $\vec k, \lambda$\\ \\
Hamiltonoperator:
\begin{eqnarray*}
H = \int d^3 r\ham = \dots = \sum_{\vec k, \lambda} \frac{c\hbar|\vec k|}2 \left(a^\dagger_{\vec k \lambda}a_{\vec k \lambda}+a_{\vec k \lambda}a_{\vec k \lambda}^\dagger \ri)
\end{eqnarray*}
Mit bosonischen Vertauschungsrelationen gilt: $aa^\dagger = 1 + a^\dagger a $ und damit
\begin{eqnarray*} 
	H = \sum \limits_{\vec k \lambda}\hbar \omega_{\vec k}\left(a_{\vec k \lambda}^\dagger a_{\vec k \lambda}+\frac 1 2\ri)
\end{eqnarray*}
Das elektromagnetische Feld ist in quantisierter Form äquivalent zu Normalmoden harmonischer Oszillatoren (die Amplitude entspricht gerade den Feldern).

\subsubsection{Feldoperatoren}
\begin{eqnarray*}
	A^{\mu}_{\lambda k}(x)&=& \varepsilon_{\lambda}^{\mu} \phi_{\vec{k}}(x) \left( e^{-i\omega t} a_{k\lambda} +e^{i\omega t} a^\dagger_{k\lambda} \right)
\\
\Rightarrow \vec E &=& -\dot{\vec A} = -i\omega \varepsilon_{\lambda}\phi_k(x) \left( e^{-i\omega t} a_{k\lambda} +e^{i\omega t} a^\dagger_{k\lambda} \right)
\end{eqnarray*}
\begin{itemize}
\item Feldoperatoren sind hermitesch $\vec E^\dagger = \vec E, \ \vec A^\dagger = \vec A$ und sind daher physikalische Observablen und müssen somit Erzeuger $a^\dagger$ und Vernichter $a$ enthalten.
\begin{eqnarray*}
\left [E^i(\vec r, t),E^j(\vec r',t)\ri] \g 0\\
\left[E^i(\vec r, t), B^j(\vec r, t)\ri] \g i\varepsilon_{ijk}\partial_k\delta(\vec r-\vec r')
\end{eqnarray*}
\item Damit verhalten sich die Feldoperatoren $\vec{E}$ und $\vec{B}$ wie Impuls und Amplitude eines quantenmechanischen harmonischen Oszillators. 
\item Die Fockzustände $\ket{n}$ sind keine Eigenzustände von $\vec{E}$ und $\vec{B}$ und es gibt Fluktuationen (Unschärfen) selbst im Vakuum: 
\begin{eqnarray*}
	\bracket{0}{(\hat a+\hat a^\dagger)}{0}&=&0
	\\
	\bracket{0}{(\hat a+\hat a^\dagger)}{0}&\neq&0
\end{eqnarray*}
\end{itemize}


\subsubsection{Strahlungsübergänge, QED}
Die Kopplung vom elektromagnetischen Feld an Elektronen erfolgt nach dem Prinzip der minimalen Kopplung wie folgt: 
\begin{eqnarray*}
&& \vec p \to \vec p-e\vec A
\\
\Rightarrow& \wh\ham =& \wh\ham_{el} + \wh\ham_{el-ph} + \wh\ham_{ph}
\\
&=& \sum_\sigma \int \mathrm{d}^3r \; \hat{\Psi}_\sigma^\dagger(\vec r) \left(\frac{1}{2m} \left(\hat{\vec{p}}-e\hat{\vec A}(\vec r)\ri)^2 + e\phi(\vec r)\ri) \hat{\Phi}_\sigma(\vec r) + H_{ph}(\hat{\vec{A}})
\\
& \wh\ham_{ph} =& \sum_{\vec k \lambda}\hbar \omega_{\vec k} \hat{a}^\dagger_{\vec k \lambda}\hat{a}_{\vec k \lambda}
\\
& \wh\ham_{el-ph} =& -\frac 1 m \sum_\sigma \int\mathrm{d}^3 r\; \hat{ \Psi}_\sigma^\dagger(\vec r) \hat{\vec A}(\vec{r})\hat{\vec{p}} \hat{\Psi}_\sigma(\vec{r}) + \mathcal{O}(\vec A^2)
\end{eqnarray*}
Strahlungsübergänge folgen in zeitabhängiger Störungstheorie in $\wh\ham_{el-ph}$.
\begin{eqnarray*}
\Gamma_{i-f} = \frac{2 \pi}\hbar \delta(E_f-E_i) |\bra{f}\wh\ham_{el-ph}\ket{i}|^2
\end{eqnarray*}
Hierbei ist $\ket{i/f}=\ket{\psi_{i/f}}_{el}\otimes \ket{\psi_{i/f}}_{ph}$, das heißt es müssen Übergänge der Elektronen und Photonen beachtet werden.

Im klassischen Fall ergibt sich nach Fermis Goldener-Regel (Kapitel 1.2) die Übergangsrate $\Gamma=\dot{P}_{if}$ zu: 
\begin{eqnarray*}
\Gamma^{\text{abs / em}}\propto |\vec{E}_{\omega}|^2\delta(\varepsilon_f-\varepsilon_i \pm \hbar \omega)\end{eqnarray*}
Wobei $|\vec E_{\omega}|^2\propto n_{\omega}$ die Feldenergie ist, welche der Besetzung entspricht. 

In der Quanten Elektro Dynamik ergibt sich die Übergangsrate hingegen wie folgt: 
\begin{eqnarray*}
\bracket{\psi_i}{ (\hat{a}+\hat{a}^\dagger)}{\psi_f} &=&  \bracket{n\pm1}{(\hat{a}+\hat{a}^\dagger)}{n} = \bra{n\pm1}\big(\sqrt{n} \ket{n-1}+ \sqrt{n+1} \ket{n+1}\big)
\\
&=& \Bigg \{ \begin{array}{llc} \sqrt{n} & \text{für Absorption} & (-) \\ \sqrt{n+1} & \text{für Emission}& (+) \end{array}
\\
\Rightarrow\quad \Gamma &\propto& \Bigg \{ \begin{array}{ll} n &\text{für Absorption}\\ n+1 &\text{für stimulierte und spontane Emission }\end{array}
\end{eqnarray*}
\begin{itemize}
\item Absorption und stimulierte Emission ergeben klassische Raten $\propto n\propto |\vec{E}|^2$
\item Spontane Emission ($n=0$) ins Vakuum erfolgt ohne äußeres Feld, verursacht durch sogenannte \underline{Vakuumsfluktuation} und ist eine reine Folgerung aus der QED. 
\end{itemize}


\subsubsection{Der Casimir-Effekt}
Bisher haben wir die Vakuumenergie vernachlässigt, weil sie nicht beobachtbar ist. Vakuumsfluktuationen der Feldamplituden sind aber nach der QED vorhanden.

Dies Vakuumenergie hängt von den Randbedingungen ab. Damit sollte sie messbar sein. Wir betrachten als Beispiel zwei leitende Platten im Abstand $a$. Die Moden sind dann gegeben als: 
\begin{eqnarray*}
\omega(\vec k_\parallel , n) = c \sqrt{|k_\parallel|^2 + \left(\frac{\pi n }a\ri)^2}\qquad n = 1,2,\dots
\end{eqnarray*}
Wegen der Randbedingungen sei $E_\perp = 0$ auf den Platten. Damit ergibt sich eine unterschiedliche Energie im Vakuumzustand mit Platten $E$ und ohne die Platten $E_0$: 
\begin{eqnarray*}
E = \sum_{\vec k \lambda}\frac 1 2 \hbar \omega_{\vec k} = \frac\hbar 2 \int \frac{L^2 \mathrm{d}^2k_\parallel}{(2\pi)^2} \sum_{n=1}^\infty 2 \omega ( \vec k_\parallel,n) &\quad& E_0 = \frac{\hbar c}2 \int \frac{L^2 \mathrm{d}^2k_\parallel}{(2 \pi)^2} \int_0^\infty \mathrm{d}n \omega(\vec k_\parallel, n)
\end{eqnarray*}
Die Energie pro Fläche $\varepsilon$ und die Kraft pro Fläche $F/L^2$ ergeben sich damit zu: 
\begin{eqnarray*}
\varepsilon &=& \frac{E-E_0}{L^2} = \dots = - \frac{\pi^2}{720}\frac{\hbar c}{a^3}
\\
\frac F{L^2} &=& - \frac{\partial \varepsilon}{\partial a} = - \frac{\pi^2}{240}\frac{\hbar c}{a^4}
\end{eqnarray*}
Dieser Effekt wird als Casimir-Effekt bezeichnet. Folgende Anmerkungen 
\begin{itemize}
\item Die anziehende Kraft ist sehr klein, für $L=\unit{1}{\meter}$ und $a=\unit{1}{\micro\meter}$ ergibt sich beispielsweise zu $F\approx \unit{10^2}{\newton}$. 
\item Die Kraft hängt nur von Naturkonstanten und vom Abstand ab, aber nicht vom Material.
\item Der Effekt kann kann auf andere Geometrien, beispielsweise Kugeln, erweitert werden und hängt von der Temperatur ab. 
\item Der Effekt ist ähnlich der Betrachtung von van-der-Waals-Kraft zwischen nicht fluktuierenden Dipolen in Molekülen oder Atomen. 
\end{itemize}
